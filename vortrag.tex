\documentclass[11pt]{beamer}
\usetheme{Berlin}
\usepackage[utf8]{inputenc}
\usepackage[german]{babel}
\usepackage[T1]{fontenc}
\usepackage{amsmath}
\usepackage{amsfonts}
\usepackage{amssymb}
\author{Patrik Bütler, Daniel Kurman, Émile Schneider}
\title{Tote Hardware zum Leben erwecken}
%\setbeamercovered{transparent} 
%\setbeamertemplate{navigation symbols}{} 
%\logo{} 
%\institute{} 
%\date{} 
%\subject{} 
\begin{document}

\begin{frame}
\titlepage
\end{frame}

\begin{frame}
\frametitle{Vorhaben}

\begin{itemize}
 \item<1-> ein einfaches Operating System zu entwickeln
 \item<2-> möglichst Hardwarenah ohne auf Voodoo Black Magic vertrauen zu müssen
\item<3-> erstes Ziel: Samsung Galaxy S3 mit Exynoschip
\item<4-> alternativ: Raspberry Pi
\end{itemize}
\end{frame}



\begin{frame}{•}
\frametitle{Motivation}
\begin{itemize}
\item<1-> Hersteller verhindert das Booten eines OS das kein Androidderivat ist
\item<2-> auch konnen wir nicht direkt mit der Hardware kommunizieren
\item<3-> alles lauft hinter einer Trustzone-API
\end{itemize}
\end{frame}
\begin{frame}
\frametitle{Ziel}
\begin{itemize}
\item<1-> Absolute Macht
\item<2-> Plan: A
	\begin{itemize}
	\item<3->
	\end{itemize}
\end{itemize}

\end{frame}
\end{document}